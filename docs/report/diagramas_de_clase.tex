A continuación presentamos algunos diagramas de clase que muestran on algo más de
detalle las entidades anteriormente descriptas

\subsection{Clase Pokemon}
  \def\scale{.6}
  \def\path{Pokemon.png}
  \def\text{Pokemon}
  \input{report/image.tex}

\subsection{Clase Ataque}
  \def\scale{.6}
  \def\path{Ataque.png}
  \def\text{Ataque}
  \input{report/image.tex}

\subsection{Clase Elemento}
  \def\scale{.6}
  \def\path{Elemento.png}
  \def\text{Elemento}
  \input{report/image.tex}

\subsection{Clase Ataque}
  \def\scale{.6}
  \def\path{Ataque.png}
  \def\text{Ataque}
  \input{report/image.tex}

\subsection{Clase Entrenador}
  \def\scale{.6}
  \def\path{Entrenador.png}
  \def\text{Entrenador}
  \input{report/image.tex}

\subsection{Clase Modo}
  \def\scale{.6}
  \def\path{Modo.png}
  \def\text{Modo}
  \input{report/image.tex}
  Son herederas de Modo las clases:
  \begin{itemize}
    \item ModoInhabilitadoPor1Turno
    \item ModoInhabilitadoPor2Turnos
    \item ModoInhabilitadoPor3Turnos
    \item ModoDañoPermanente
    \item ModoDañoPermanenteEInhabilitadoPor1Turno
    \item ModoDañoPermanenteEInhabilitadoPor2Turnos
    \item ModoDañoPermanenteEInhabilitadoPor3Turnos
    \item ModoDebilitado
  \end{itemize}

\subsection{Clase Juego}
  \def\scale{.6}
  \def\path{Juego.png}
  \def\text{Juego}
  \input{report/image.tex}