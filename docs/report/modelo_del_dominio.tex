\subsection{Entidades del modelo}
  \begin{itemize}
    \item \underline{\textbf{Pokemon:}} modela a un Pokémon genérico capaz de atacar y recibir ataques.
    \item \underline{\textbf{Subtipos de Pokemon:}} son clases que heredan de Pokémon y modelan a los
    distintos tipos que plantea la consigna: Fuego,Agua,Hierba y Normal. Cada subtipo tiene redefinidos
    los métodos recibirAtaque correspondientes para lograr representar la ventaja de tipos
    \item \underline{\textbf{Pokemones concretos:}} representan a los Pokémon que el usuario final de la
    aplicación podrá utilizer para pelear
    \item \underline{\textbf{Ataque:}} modelan las técnicas de combate de los Pokémon. Se ejecutan con el
    método correspondiente y le mandan un mensaje al objetivo para que decremente sus puntos de vida según
    corresponda.
    \item \underline{\textbf{Subtipos de ataque:}} modelan los distintos tipos de ataques que contempla la
    consigna: Fuego,Agua,Hierba y Normal.
    \item \underline{\textbf{Ataques concretos:}} modelan los ataques que el usaurio final podrá ordenar
    realizar a los Pokémon.
    \item \underline{\textbf{Entrenador:}} modela al jugador que participa de la batalla. Tiene un conjunto
    de Pokémon y de elementos a su disposición. Posee además los métodos para hacer uso de dichos elementos,
    darle órdenes de ataque a sus Pokémon o hacer cambios dentro del equipo.
    \item \underline{\textbf{Elemento:}} esta clase modela a los elementos que un entrenador puede usar
    durante la batalla. Tienen diversos efectos como recuperar puntos de vida o curar al Pokémon de
    estados adversos.
    \item \underline{\textbf{Modo:}} esta clase modela el estado en que se encuentra un Pokémon. Heredan
    de esta clase, entre otras, las clases ModoDañoPermanente y ModoInhabilitadoPor1Turno, las cuales se
    encargan de modelar estados adversos. En el caso de ModoDañoPermanente , por ejemplo, el comportamiento
    del Pokémon es modificado de manera que, en cada turno, pierde una cantidad de puntos de vida equivalente
    al 10 \% de sus puntos de vida originales.
    \item \underline{\textbf{Juego:}} esta clase es la encargada de interactuar con la interfaz gráfica
    recibiendo las órdenes que ésta le pase y ejecutándolas en el modelo.
  \end{itemize}