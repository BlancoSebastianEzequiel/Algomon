Implementamos los métodos de un Pokémon genérico aplicando el patrón double dispatch.
Todas las instancias de la clase Pokémon tienen métodos para recibir ataques de diversos tipos pero,
dependiendo de la subclase a la que pertenezca cada instancia, el efecto de los ataques que se
reciban varía. \\

A su vez, todos los Pokémon pueden lanzar ataques (recibiendo previamente uan instancia del ataque
en cuestión y un objetivo como parámetro). Dependiendo del subtipo de la clase \code{Ataque} al que pertenezca la
instancia recibida, ejecutar el método atacar resultará en enviarle un mensaje distinto al enemigo.
Atacar con \code{Brasas}, por ejemplo, resultará en enviarle el mensaje \code{recibirAtaqueDeFuego()} al enemigo.
Por ejemplo, Squirtle y Charmander tienen el método \code{recibirAtaqueDeHierba()} al igual que todo Pokémon. \\
Pero \code{Squirtle} pertenece además al subtipo \code{WaterTypePokemon}, por lo que \code{recibirAtaqueDeHierba()}
le debitará a sus puntos de vida el doble de la potencia del ataque recibido. Por otra parte \code{Charmander}
pertenece al subtipo \code{FireTypePokemon}, lo que significa que tiene el método \code{recibirAtaqueDeHierba()}
implementado de manera diferente. En cuestión, lo tiene implementado para debitar de sus puntos de
vida solamente la mitad de la potencia del ataque recibido. De esta manera, siguiendo este patrón
podemos modelar el sistema de ventaja de tipos entre Pokémon. \\

Los Pokémon también pueden sufrir estados adversos que modifiquen su comportamiento durante la batalla.
Para eso tienen un atributo accionador que lleva dentro una instancia de Modo. De esta manera, algunos
de los ataques que reciban pueden cambiar el Modo que llevan por Inhabilitado o quemado por ejemplo.

Estos Pokémon serán comandados luego por un entrenador que llevará tres instancias dentro de su
aributo equipoPokémon. \\

Por encima de los entrenadores está la clase juego, que lleva como atributo dos instancias de entrenador,
es decir, a los dos jugadores. \\

El juego recibirá las órdenes del jugador a través de la interfaz gráfica y le ordenará al entrenador
del turno actual ejecutar dicha acción. Entre una acción y la otra el juego irá cambiando los turnos
para que cada jugador pueda jugar.