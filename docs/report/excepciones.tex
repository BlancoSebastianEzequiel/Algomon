Las excepciones que contempla el modelo son las siguientes:
\begin{itemize}
  \item \underline{\textbf{NoPuedeRealizarElAtaqueException:}} Esta excepción se lanza cuando un Pokémon
  intenta efectuar un ataque que ya consumió por completo. La excepción es atrapada por el método atacar
  del mismo Pokémon y, de esta manera, conseguimos que un ataque no se puede usar más veces de las
  contempladas por la consigna.
  \item \underline{\textbf{PokemonSeDebilitoException:}} Esta excepción es lanzada cuando, producto de un
  ataque, un Pokémon se queda sin puntos de vida. La excepción es lanzada y asciende hasta llegar al
  entrenador que dio la orden de ataque que resultó en el debilitamiento de dicho Pokémon. El
  entrenador atrapa la excepción y se suma un punto a sun puntaje (pues derrotó a uno de los Pokémon
  del enemigo). Esta excepción se lanza en otros ámbitos también como, por ejemplo, cuando un entrenador
  intenta cambiar a su Pokémon actual por otro de su equipo que ya no tiene puntos de vida. En este caso
  la misma interfaz se encarga de atrapar la excepción y realizar las acciones que correspondan.
  \item \underline{\textbf{EquipoCompletoException:}} Esta excepción se lanza cuando un entrenador
  intenta agregar a su equipo de tres Pokémon un cuarto. Esta excepción es atrapada luego por la interfaz
  que se encarga de realizar las acciones que correspondan.
  \item \underline{\textbf{GanadorException:}} Cuando un entrenador se anota un tercer punto a su
  puntaje, significa que derrotó a los tres Pokémon del enemigo. Entonces lanza esta excepción que
  luego la interfaz captura declarando ganador al entrenador que lanzó esta excepción.
\end{itemize}